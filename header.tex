\documentclass[utf8,russian,a4paper,14pt,%
    simple,floatsection,equationsection]{eskdtext}

%\usepackage{mathtext}       % Русские буквы в формулах
\usepackage{amsmath}        % Русские буквы в формулах
\usepackage{caption}        % Позволит подправить заголовки таблиц
\usepackage{longtable}      % Таблицы с переносом на след. страницу 
\usepackage{tabularx}
\usepackage{graphicx}       % Используем графику в документе
\usepackage{textcase}       % Заголовки в ВЕРХНЕМ РЕГИСТРЕ
\usepackage{rotating}       % Поворот текста
\usepackage{eskdtotal}      % Счетчики
\usepackage[T2A]{fontenc}
\usepackage{enumitem}
\setlist{noitemsep,nolistsep}

% Двухсантиметровый отступ слева перед заголовками таблиц
\captionsetup[table]{%
    format=hang,labelsep=dash,justification=raggedright,singlelinecheck=false,%
    aboveskip=3mm,belowskip=0mm,margin=2cm}

% Настойка заголовков разделов и подразделов ESKDX
\ESKDsectAlign{section}{Center}
\ESKDsectStyle{section}{\MakeTextUppercase}
\ESKDsectSkip{section}{5mm}{7mm}

\ESKDsectAlign{subsection}{Left}
\ESKDsectStyle{subsection}{}
\ESKDsectSkip{subsubsection}{2cm plus 0pt minus 0pt}{2cm plus 0pt minus 0pt}

\ESKDsectAlign{subsubsection}{Left}
\ESKDsectStyle{subsubsection}{}
%\ESKDsectSkip{subsubsection}{5mm}{7mm}

% избавляемся от курсива в формулах
%\DeclareSymbolFont{letters}{OT1}{cmr}{m}{n}
\everymath={\rm}
\everydisplay={\rm}

\usepackage{tocloft}
\renewcommand{\cftparskip}{-1mm}

\renewcommand{\cfttoctitlefont}{\hspace{0.38\textwidth} \MakeUppercase}
\renewcommand{\cftbeforetoctitleskip}{-1em}

\renewcommand{\cftsecnumwidth}{1em}
\renewcommand{\cftsubsecnumwidth}{1,7em}
\renewcommand{\cftsubsubsecnumwidth}{2,5em}

\renewcommand{\cftsecindent}{0em}
\renewcommand{\cftsubsecindent}{0em}
\renewcommand{\cftsubsubsecindent}{0em}

\renewcommand{\cftsecpagefont}{\normalsize}

\renewcommand{\cftsecfont}{\normalsize}
\renewcommand{\cftsubsecfont}{\normalsize}
\renewcommand{\cftsubsubsecfont}{\normalsize}

\renewcommand{\cftbeforesecskip}{0em}

\renewcommand{\cftdotsep}{1}
\renewcommand{\cftsecdotsep}{1}
\setcounter{tocdepth}{3} % задать глубину оглавления — до subsubsection включительно

