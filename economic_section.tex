\section{Экономическая часть}
    \subsection{Расчет капитальных затрат}
        В данном проекте проводилось проектирование и разработка автономного
        инвертора напряжения с реализацией закона поддержания постоянства
        соотношения U/f для управления асинхронным электроприводом.

        Спроектированный инвертор напряжения позволит проводить дальнейшие
        исследования и построение системы управления для демпфирования упругого
        соединения двигателя с инерционной массой. 

        В данном разделе приведен сравнительный расчет экономической
        эффективности разработанного инвертора напряжения с инвертором
        напряжении фирмы MOELLER. Следует учитывать также то, что при
        возможности серийного выпуска продукции ее себестоимость будет
        уменьшаться в зависимости от количества выпускаемой партии.

        Оценим экономическую целесообразность постройки стенда
        электромеханической системы асинхронного электропривода на основе
        разработанного частотного преобразователя, при этом в качестве базового
        варианта возьмем аналогичный по мощности частотный преобразователь
        MMX12 - 0,75 кВт фирмы MOELLER.

        В состав капитальных затрат системы электропривода входят: стоимость
        оборудования системы, стоимость резерва, строительно-монтажных работ, в
        том числе и заработная плата, транспортные расходы доставки
        оборудования, стоимость занимаемой площади здания,
        заготовительно-складские затраты.

        Определим стоимость оборудования установки базового варианта \\(таблица
        \ref{table:cost-base}).

        Определим стоимость оборудования установки нового варианта \\(таблица
        \ref{table:cost-new}).

        Стоимость резерва системы составляет 30\% от стоимости основного
        оборудования. Затраты на площадь помещения, где расположено
        оборудование, транспортные расходы, и заготовительно-складские затраты,
        принимают соответственно 15, 4 и 1,2\% от стоимости основного
        оборудования. Стоимость строительно-монтажных работ для данной системы
        составляет 10\% от стоимости основного оборудования (50\% этой суммы
        составляет заработная плата).

        \begin{longtable}{|p{8cm}|p{2,5cm}|p{2,5cm}|p{2,5cm}|}
        \caption{Стоимость оборудования
            базовой системы \label{table:cost-base}}\\
        \hline
        Наименование & Стоимость & Количество & Общая стоимость\\
        %\hline
        %1 & 2 & 3 & 4 \\
        \hline
        \endfirsthead
        \caption*{Продолжение таблицы
            \ref{table:cost-base}}\\
        \hline
        \endhead
        Частотный преобразователь MOELLER MX12 & 2700 грн & 1 шт & 2700 грн\\
        \hline
        Двигатель асинхронный АИР56А4У3 & 600 грн & 1 шт & 600 грн\\
        \hline
        Датчик положения ПДФ-5 & 500 грн & 2 шт & 1000 грн\\
        \hline
        \multicolumn{3}{|l|}{Общая сумма затрат на компоненты} & 4300 грн\\
        \hline
        \end{longtable}

        \begin{longtable}{|p{8cm}|p{2,5cm}|p{2,5cm}|p{2,5cm}|}
        \caption{Стоимость оборудования
            новой системы \label{table:cost-new}}\\
        \hline
        Наименование & Стоимость & Количество & Общая стоимость\\
        \hline
        1 & 2 & 3 & 4 \\
        \hline
        \endfirsthead
        \caption*{Продолжение таблицы
            \ref{table:cost-new}}\\
        \hline
        1 & 2 & 3 & 4 \\
        \endhead
        Конденсатор 10 мкФ 20 В & 0,20 грн & 5 шт. & 1 грн \\
        \hline
        Конденсатор 1000 мкФ 25 В & 1 грн & 1 шт. & 1 грн \\
        \hline
        Конденсатор 0,1 мкФ 50 В & 0,3 грн & 4 шт. & 1,2 грн \\
        \hline
        Конденсатор 0,01 мкФ 50 В & 0,3 грн & 6 шт. & 1,8 грн \\
        \hline
        Конденсатор 330 мкФ 450 В & 35 грн & 2 шт. & 70 грн \\
        \hline
        Реле 12V DC 16 A & 10 грн & 1 шт. & 10 грн \\
        \hline
        Резистор 0,25 Вт & 0,2 грн & 21 шт. & 4,2 грн \\
        \hline
        Резистор SMD 1206 & 0,1 грн & 35 шт. & 3,5 грн \\
        \hline
        Резистор 5 Вт & 1 грн & 2 шт. & 2 грн \\
        \hline
        Транзистор IRF740 & 6,5 грн & 7 шт. & 46 грн \\
        \hline
        ИМС IR2130 & 50 грн & 1 шт. & 50 грн \\
        \hline
        Диод FR106 & 0,2 грн & 10 шт. & 2 грн \\
        \hline
        Диод 1n4148 & 0,15 грн & 6 шт. & 0,9 грн \\
        \hline
        Стабилитрон 15 В 0,5 Вт & 0,5 грн & 6 шт & 3 грн \\
        \hline
        Мост диодный 35 А 1000 В & 15 грн & 1 шт & 15 грн \\
        \hline
        ИМС ADUM1300 & 35 грн & 3 шт & 105 грн \\
        \hline
        ИМС ADUM1201 & 17 грн & 1 шт & 17 грн \\
        \hline
        ИМС AT90PWM3 & 30 грн & 1шт & 40 грн \\
        \hline
        ИМС 74HC04 & 5 грн & 1шт & 5 грн \\
        \hline
        Провод монтажный & & & 15 грн\\
        \hline
        Стеклотекстолит фольгированный & & & 35 грн\\
        \hline
        Теплоотводящий радиатор  & 25 грн & 1 шт & 25 грн\\
        \hline
        Двигатель асинхронный АИР56А4У3 & 600 грн & 1 шт & 600 грн\\
        \hline
        Датчик положения ПДФ-5 & 500 грн & 2 шт & 1000 грн\\
        \hline
        \multicolumn{3}{|l|}{Общая сумма затрат на компоненты} & 2100 грн\\
        \hline
        \end{longtable}

        Рассчитаем капитальные затраты на оборудование для обоих вариантов:
        общая сумма оборудования базовой системы $Ц_{об} = 4300 \; \text{грн}$,
        модернизированной системы $Ц_{об} = 2060 \; \text{грн}$.

        Определим затраты на строительно-монтажные работы
        \begin{equation}
            S_\text{смр} = \text{Ц}_\text{об} \cdot 0,1.
        \end{equation}

        Определим заработную плату строительно-монтажных рабочих
        \begin{equation}
            S_\text{см} = S_\text{смр} \cdot 0,5.
        \end{equation}

        Определим общие затраты на оборудование
        \begin{equation}
            S_\text{об} = \text{Ц}_\text{об} + S_\text{см}. 
        \end{equation}

        Определим стоимость резерва системы
        \begin{equation}
            S_\text{рез} = \text{Ц}_\text{об} \cdot 0,3.
        \end{equation}

        Определим стоимость площади, занимаемую оборудованием
        \begin{equation}
            S_\text{пл} = \text{Ц}_\text{об} \cdot 0,15.
        \end{equation}

        Рассчитаем транспортные расходы на доставку оборудования
        \begin{equation}
            S_\text{тр} = \text{Ц}_\text{об} \cdot 0,04.
        \end{equation}

        Рассчитаем заготовительно-складские затраты
        \begin{equation}
            S_\text{зс} = \text{Ц}_{об} \cdot 0,012. 
        \end{equation}

        Рассчитаем общую сумму капитальных затрат
        \begin{equation}
            К = S_\text{об} + S_\text{рез} +
                S_\text{пл} + S_\text{тр} + S_\text{зс}. 
        \end{equation}

        Все расчеты проведем по вышеприведенным формулам и запишем в таблицу
        \ref{table:capital-cost}.

        \begin{longtable}{|p{8cm}|p{3cm}|p{4cm}|}
            \caption{Капитальные затраты на оборудование
                \label{table:capital-cost}}\\
            \hline
            Затраты & Базовая система & Модернизирован- ная система\\
            \hline
            \endfirsthead
            \caption*{Продолжение таблицы \ref{table:capital-cost}}\\
            \hline
            Затраты & Базовая система & Модернизирован- ная система\\
            \endhead
            \hline
            зарплата строительно-монтажных рабочих, грн & 215 & 105\\
            \hline
            затраты на оборудование, грн & 4515 & 2205\\
            \hline
            стоимость резерва, грн & 1290 & 630\\
            \hline
            стоимость площади, грн & 645 & 315\\
            \hline
            транспортные расходы, грн & 172 & 84\\
            \hline
            заготовительно-складские затраты, грн & 51,6 & 25,2\\
            \hline
            сумма капитальных затрат, грн & 6673,6 & 3259,2\\
            \hline
        \end{longtable}

    \subsection{Расчет эксплуатационных затрат}

        Эксплуатационные затраты определяются себестоимостью, которая состоит
        из:
        \begin{itemize}
            \item амортизационные отчисления;
            \item затраты на потребленную электроэнергию;
            \item затраты на ремонт оборудования;
            \item другие затраты.
        \end{itemize}

        Примем усредненную норму амортизации 8\% для всех объектов.

        Определим амортизационные отчисления
        \begin{equation}
            C_\text{а} = N_\text{а} \cdot \text{Ц}_\text{об}.
        \end{equation}

        Рассчитаем отчисления за площадь
        \begin{equation}
            C_\text{пл} = N_\text{а} \cdot S_\text{пл}.
        \end{equation}

        Рассчитаем полные амортизационные отчисления
        \begin{equation}
            C_\text{аПОЛ}  = C_\text{а} + C_\text{пл}.
        \end{equation}

        Проведем расчет эффективного фонда времени, при работе 12 часов в
        день, за год на складе
        \begin{equation}
            T_\text{эф} = 12 \cdot 365 = 4380 \; \text{ч}. 
        \end{equation}

        Рассчитаем расходы на потребляемую электроэнергию
        \begin{equation}
            C_\text{э} = \frac{P}{\eta} \cdot
                T_\text{эф} \cdot K_\text{в} \cdot K_\text{м} \cdot c ,
        \end{equation}

        где  $P$ – номинальная мощность используемого
            электродвигателя, кВт;\par
        $\eta$ – коэффициент полезного действия электрооборудования,
            доли;\par
        $T_\text{эф}$ – эффективный фонд времени работы, ч;\par
        $K_\text{в}$ – коэффициент использования по времени;\par
        $K_\text{м}$ – коэффициент использования по мощности;\par
        $C_\text{э}$ – стоимость 1 кВт ч электроэнергии, грн/кВт ч.\par

        Коэффициент полезного действия электрооборудования вычисляем как
        произведение коэффициентов полезного действия двигателя и
        преобразователя.  Для базового и нового вариантов коэффициент полезного
        действия равен 85\%.

        Коэффициент использования по времени для базового варианта равен 0,8.
        Коэффициент использования по мощности 0,67.

        Стоимость электроэнергии равна 0,45 грн/кВт ч.

        Эффективный фонд времени по обоим вариантам при работе в одну смену за
        год составит
        \begin{equation}
            T_\text{эф} = 8 \cdot 22 \cdot 12 = 2112 \; \text{ч}. 
        \end{equation}

        Рассчитаем затраты на текущий ремонт.

        Текущий ремонт электрооборудования производится на месте его установки
        с отключением от сети и остановкой силами сменного ремонтного
        персонала, обслуживающего данный агрегат (оборудование).  Затраты на
        текущий ремонт электрооборудования содержат следующие статьи:

        \begin{itemize}
            \item основная зарплата рабочих с начислениями $C_\text{зп}$;
            \item стоимость используемых материалов и комплектующих изделий
                $C_\text{м}$;
            \item цеховые и общезаводские расходы $C_\text{общ}$.
        \end{itemize}

        Для определения зарплаты рабочих-ремонтников необходимо знать
        трудоемкость ремонтных работ и эффективный фонд времени одного
        рабочего.  Трудоемкость ремонтных работ определяют из графика
        планово–предупредительных ремонтов.  Трудоемкость ремонтных работ
        составляет 25 чел–ч.  Эффективный фонд времени одного рабочего состоит
        из дней, оставшихся после вычитания из 365 календарных дней выходных и
        праздничных дней, а также дней, касающихся прочих невыходов на работу.
        Занятость по времени – 0,96. Эффективный фонд времени равен:
        \begin{equation}
            T = 8 \cdot (365 - 104) \cdot 0,96 = 2004,5 \; \text{ч}. 
        \end{equation}

        Заработная плата определяется через трудоемкость ремонтов и тарифную
        часовую ставку электромонтера, которая составляет 8 грн/ч.

        Тарифная зарплата
        \begin{equation}
            C_\text{зп}^T = 8 \cdot 25 = 200 \; \text{грн}. 
        \end{equation}

        К начислениям зарплаты относят премии (20\% от тарифной зарплаты),
        дополнительная зарплата (10\% от тарифной зарплаты), другие доплаты
        (10\% от тарифной зарплаты). В итоге начисления достигают 40\% от
        тарифной зарплаты.  Чтобы определить полную сумму выплат по зарплате
        рабочим, необходимо тарифную зарплату умножить на коэффициент 1,4.

        Таким образом, сумма полных выплат по зарплате
        \begin{equation}
            C_\text{зп} = C_\text{зп}^T \cdot 1,4,
        \end{equation}

        Затраты на материал и комплектующие изделия составляют
        \begin{itemize}
            \item при капитальном ремонте – 50\% от тарифной зарплаты;
            \item при среднем ремонте – 35\% от тарифной зарплаты;
            \item при текущем ремонте – 15\% от тарифной зарплаты.
        \end{itemize}

        В базовом варианте предусмотрены один капитальный ремонт, один средний
        и два текущих.

        Затраты на материалы и комплектующие будут равны
        \begin{equation}
            C_\text{м} = C_\text{зп}^T. 
        \end{equation}

        В смете годовых эксплуатационных расходов прочие расходы принимаются в
        размере 1\% от суммы капитальных вложений:
        \begin{equation}
            C_\text{пр.баз} = 0,01 \cdot K.
        \end{equation}

        Для анализа полученные данные запишем в таблицу (таблица
        \ref{table:usage-cost}).

        \begin{longtable}{|p{8cm}|p{3cm}|p{4cm}|}
            \caption{Эксплуатационные затраты
                \label{table:usage-cost}}\\
            \hline
            Затраты & Базовая система & Модернизирован- ная система\\
            \hline
            \endfirsthead
            \caption*{Продолжение таблицы \ref{table:usage-cost}}\\
            \hline
            Затраты & Базовая система & Модернизирован- ная система\\
            \endhead
            \hline
            Амортизационные отчисления, грн & 395,6 & 193\\
            \hline
            Затраты на электроэнергию, грн & 71 & 71\\
            \hline
            Заработная плата, грн & 280 & 280\\
            \hline
            Расходы на материалы, грн & 48 & 28\\
            \hline
            Общие расходы, грн & 143,5 & 117\\
            \hline
            Другие расходы, грн & 68 & 32\\
            \hline
            Всего эксплуатационных расходов, грн & 661 & 553\\
            \hline
        \end{longtable}
    
    \subsection{Расчет эффективности проектируемой системы}
        Так как величины капитальных вложений и эксплуатационных расходов при
        внедрении новой (усовершенствованной) системы электропривода станка
        стали меньше, чем при базовой системе, то для определения эффективности
        и целесообразности производимых изменений следует рассчитать
        сравнительные показатели.  
        
        Для сравнения эксплуатационных затрат используем показатель
        относительной экономии (уменьшения) затрат
        \begin{equation*}
            \lambda_\text{э} = 
                \frac{\text{Э}_\text{баз} - \text{Э}_\text{нов}}
                    {\text{Э}_\text{баз}} \cdot 100 \% = 
                        \frac{661-533}{661} \cdot 100 \% \approx 19 \%.
        \end{equation*}

        Приведенные затраты по базовому варианту составили
        \begin{equation*}
            \text{З}_\text{пр.баз} = 
                \text{Э}_\text{баз} + E_\text{н} \cdot K_\text{баз} = 
                    661 + 0,15 \cdot 6673,6 = 1662,04 \; \text{грн}.
        \end{equation*}

        По новому варианту
        \begin{equation*}
            \text{З}_\text{пр.нов} = 
                \text{Э}_\text{нов} + E_\text{н} \cdot K_\text{нов} = 
                    553 + 0,15 \cdot 3259,2 = 1021,88 \; \text{грн}.
        \end{equation*}

        Определим месячный экономический эффект
        \begin{equation*}
            \text{Э}_\text{м} = \text{Э}_\text{баз} 
                - \text{Э}_\text{нов} = 661 - 533 = 128 \; \text{грн}.
        \end{equation*}

        Срок окупаемости
        \begin{equation*}
            T_\text{o} = K_\text{нов}/(12 \cdot \text{Э}_\text{м})
                = 3259,2 / (12 \cdot 128) = 2,12 \; \text{года}.
        \end{equation*}

        Коэффициент эффективности капитальных вложении
        \begin{equation*}
            E = 1 / T_\text{о} = 1 / 2,12 = 0,47.
        \end{equation*}

       Расчетный коэффициент эффективности больше нормативного 
        \begin{equation*}
            E > E_\text{н},\qquad 0,47 > 0,15. 
        \end{equation*}

        Таким образом можно сделать вывод о том, что за счет более низкой
        стоимости капитальных затрат на построение частотного преобразователя в
        сравнении с затратами на использованием перобразователя выпускаемого
        промышленностью снижены эксплуатационные затраты на 19\%.
        Следовательно, по результатам вычислений, создание стенда на основе
        разработанного преобразователя экономически эффективно. Это
        подтверждается результатами технико-экономического обоснования,
        проведенного в данном дипломном проекте:
        \begin{itemize}
            \item имеем экономию капитальных вложений и эксплуатационных
                затрат;
            \item экономию электроэнергии благодаря правильному выбору ЭД в
                соответствии с режимом работы;
            \item повышение эффективности работы электродвигателя и как
                следствие снижение эксплуатационных расходов.
        \end{itemize}

