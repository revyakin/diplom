\section*{Введение}
\addcontentsline{toc}{section}{Введение}
    В настоящее время, наиболее быстрое развитие получили электроприводы двухзвенными
    преобразователями частоты, выполняемыми на основе автономных инверторов напряжения (АИН) или
    инверторов тока (АИТ). В качестве выпрямителя в таких схемах преобразователей используется либо
    неуправляемые выпрямители напряжения или тока, либо относительно новые схемы активных
    выпрямителей напряжения или тока.  Особенностью выходных напряжений АИН является пульсирующий
    характер, который обусловлен несовершенством существующих алгоритмов управления ключами АИН,
    особенно в области малых частот.  Улучшить качество выходного напряжения АИН возможно при
    уменьшении периода модуляции с использованием известных алгоритмов широтно-импульсной модуляции
    (ШИМ). Однако это уменьшение ограничено динамическими возможностями силовых полупроводниковых
    ключей и значительным ростом дополнительных коммутационных потерь.  Широтно-импульсная модуляция
    в автономном инверторе стала активно применяться вследствие появления высокопроизводительных,
    ориентированных только на задачи электропривода микроконтроллеров, которые имеют достаточный
    набор периферии. Наибольших успехов в создании таких микроконтроллеров в конце двадцатого века
    достигли такие известные мировые производители как: STM Microelectornics, Freescale, Texas
    Instruments, Siemens, Infineon. Таким образом, программная реализация перспективных алгоритмов
    управления на базе современной микропроцессорной техники предоставила ряд новых возможностей для
    построения более качественных систем управления электроприводов. 
