\section*{Перечень ссылок}
\addcontentsline{toc}{section}{Перечень ссылок}
        1	Справочник по автоматизированному электроприводу; Под ред. В.А.
        Елисеева и А.В. Шинянского. - М.: Энергоатомиздат, 1989. - 616 с.

        2	Комплектные тиристорные электроприводы: Справочник / И.Х. Евзеров,
        А.С. Горобец, Б. И. Мошкович и др.; Под ред. В.М. Перельмутер. - М.:
        Энергоатомиздат, 1988. - 319 с.

        3	AVR494: Управление асинхронным электродвигателем переменного тока
        по принципу постоянства V/f и обычного ШИМ-управления [Електронний
        ресурс] – Режим доступу:
        http://www.gaw.ru/html.cgi/txt/app/micros/avr/AVR494.htm

        4	Панкратов А. І. Системи керування електроприводами: Навч. Посібник
        з дисципліни <<Системи керування електроприводами>>. – Краматорськ: ДДМА,
        2007. – 228 с.

        5	Асинхронные двигатели серии 4А: Справочник / А.Э. Кравчик, Н.Н.
        Шлаф, В.И. Афонин. - М.: Энергоиздат, 1982. - 504 с.

        6	Задорожний Н.А. Аналіз і синтез електромеханічних систем управління
        приводом машин з пружними механічними зв'язками: навчальний посібник по
        дисципліні <<Теорія електроприводу>>. - Краматорськ:ДГМА, 2010. –192 с.

        7	Н.А. Задорожний, доцент, Н.В Климченкова, доцент, А.М. Наливайко,
        доцент. Методичні вказівки до дипломного проектування

        8	Я.Ю. Марущак Синтез електромеханічних систем з послідовним та
        паралельним коригуванням: Навч. посібник. – Львів:Видавництво
        Національного університету <<Львівська політехніка>>, 2005. –208 с.

        9	Шавьолкін О.О., Наливайко О.М. Перетворювальна техніка: навчальний
        посібник / Під загальною редакцією канд.. техн. наук,  доц.. О.О.
        Шавьолкіна. – Донецьк-Краматорськ: ДДМА, 2003. – 330 с.

        10	Задорожний Н.А. Єлементі теории єлектромеханического взаимодействия
        в двухмассовіх системах електропривода с упругими механическими
        свіязями: Учебное пособие по дисциплине <<Теория электропривода>> для
        студентов специальности <<Электромеханические системы автоматизации и
        элетропривод>> дневной формы обучения. – Часть 1. – Краматорск: ДГМА,
        2006. –72 с.

        11	Дементій Л.В., Юсіна А.Л.Охорона праці в
        автоматизованомувиробництві. Забезпечення безпеки праці – Краматорськ:
        ДДМА, 2007. – 300с.

        12	Шавьолкін О.О., Наливайко О.М. Перетворювальна техніка: навчальний
        посібник / Під загальною редакцією канд.. техн. наук,  доц.. О.О.
        Шавьолкіна. – Донецьк-Краматорськ: ДДМА, 2003. – 330 с.

        13	Методичні вказівки для курсового та дипломного проектування з
        дисципліни <<Цивільна оборона>>/Сост. Кузнецов А.А., ПоляковО. Е.,
        Гліняна Н.М., Юсіна А.Л. – Краматорськ: ДДМА, 2002. –16 с.

        14	Методичні вказівки до самостійної роботиз економічних розділівзвіту
        з переддипломної практики та дипломного проекту (для студентів
        електротехнічних спеціальностей всіх форм навчання) / Сост. Н.В.
        Клімченкова, О.М. Наливайко, О.В.Субботин. – Краматорськ: ДДМА, 2007. –
        48с

        15 Квашнин В.О., Чередник Ю.Н.. Разработка и исследование регулируемого
        асинхронного электропривода на основе преобразователя частоты  с
        широтно-импульсной модуляции.// Наукові праці Донецького національного
        технічного університету. – Донецьк

        16 Пащенко А.С., Назарько М.М., Квашнин В.О. Разработка и исследование
        стабилизированного источника питания для программатора STK 500// Вісник
        кафедри <<електротехніка>> за підсумками наукової діяльності студентів. –
        Донецьк. ДонНТУ, 2011 – С. 94 – 95

        17 Квашнин В.О., Чередник Ю.Н.. Разработка и исследование регулируемого
        асинхронного электропривода на основе преобразователя частоты  с
        широтно-импульсной модуляции.// Наукові праці Донецького національного
        технічного університету. – Донецьк: 

