\section*{Заключение}
\addcontentsline{toc}{section}{Заключение}
        В настоящем проекте решается задача построения частотного асинхронного
        электропривода с широтно-импульсной модуляцией на основе
        микроконтроллера STM32F100RBT6B для лабораторно-исследовательского
        стенда. Также было выполнено моделирование данного электропривода.

        Разработана СУЭП с частотным регулированием скорости обеспечив
        оптимальные нагрузочные диаграммы и тахограммы. Для определения
        экономической целесообразности проекта был проведен расчет
        технико-экономических показателей.

        Был проведен анализ опасных и вредных производственных факторов при
        работе на данной установке, разработаны мероприятия по улучшению
        условий труда на рабочем месте и проанализированы возможные аварийные
        ситуаций для эффективного их устранения.

        Таким образом, спроектированная система обеспечивает все требования
        предъявленные в задании. 
