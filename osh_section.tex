\section{Охрана труда}

        Объектом проектирования является лабораторно-исследовательский 

        Стенд регулируемой системы электропривода асинхронного двигателя с
        частотным управлением.

        В соответствии с ГОСТ 12.0.003-74 ССБТ «Опасные и вредные
        производственные факторы.  Классификация» все производственные факторы
        делятся на опасные и вредные факторы. Опасный производственный фактор –
        фактор, воздействие которого может привести к травме или другому
        резкому внезапному ухудшению здоровья. Вредный производственный фактор
        – фактор, воздействие которого может привести к снижению
        работоспособности, заболеванию или профессиональному заболеванию.
        Опасные и вредные производственные факторы подразделяются на 4 группы:
        физические, химические, биологические и психофизиологические. 

        Рассмотрим опасные и вредные производственные факторы, возникающие при
        эксплуатации частотного асинхронного электропривода.

        К физическим опасным производственным факторам относятся:
        \begin{itemize}
            \item движущиеся части электропривода  (вал двигателя, ременные
                передачи);
            \item нагретые поверхности оборудования;
            \item высокое напряжение в силовой электрической сети; 
            \item возможность возникновения пожаров.
        \end{itemize}

       Вредными физическими факторами являются:
        \begin{itemize}
            \item повышенный уровень электромагнитного излучения;
            \item повышенная напряженность электрического и магнитного полей;
            \item повышенная температура, высокая скорость движения воздуха
                рабочей зоны;
            \item повышенные уровни шума и вибрации.
        \end{itemize}

        К биологическим факторам относятся микроорганизмы, например,
        находящиеся в отработанной смазочно-охлаждающей жидкости. 

        Химические и психофизические факторы при обслуживании частотного
        асинхронного электропривода являются незначительными.

        Анализ показывает, что из всех перечисленных опасных и вредных
        производственных факторов наибольшее значение имеют физические, в
        частности, локальная вибрация и шум, электромагнитное излучение, а
        также возможность поражения электрическим током.

        Интенсивное шумовое воздействие на организм человека неблагоприятно
        влияет на протекание нервных процессов, способствует развитию
        утомления, изменениям в сердечно-сосудистой системе. Длительное
        воздействие вибрации высоких уровней на организм человека приводит к
        развитию преждевременного утомления, снижению производительности труда,
        росту заболеваемости и нередко к возникновению профессиональной
        патологии - вибрационной болезни.  Опасное воздействие на работающих
        могут оказывать электромагнитные поля радиочастот (60 кГц-300 ГГц) и
        электрические поля промышленной частоты (50 Гц). 

        
        Проведенный анализ опасных и вредных производственных факторов
        позволяет обосновать выбор мероприятий и средств по их недопущению.

    \subsection{Общие требования безопасности}

        Общие требования безопасности к производственным процессам установлены
        ГОСТ 12.3.002-75 ССБТ «Процессы производственные. Общие требования
        безопасности». Согласно ГОСТу безопасность процессов обеспечивается
        выполнением следующих мероприятий:

        \begin{itemize}
            \item выбор технологического процесса и режима работы;
            \item выбор производственного помещения или промышленной площадки;
            \item выбор производственного оборудования, его размещение и
                организация рабочих мест;
            \item рациональное распределение функций между человеком и
                оборудованием;
            \item выбор способов хранения и транспортировки исходных
                материалов, заготовок, полуфабрикатов, готовой продукции и
                отходов; 
            \item профессиональный отбор и обучение работающих; 
            \item включение требований безопасности в нормативно-технические
                документы.
        \end{itemize}

        При выборе технологического процесса и режима работы  учитывают: 
        \begin{itemize}
            \item наличие опасных и вредных производственных факторов,
                возможность их устранения и защиты от них;
            \item возможность механизации и автоматизации производства,
                применения дистанционного управления;
            \item внедрение систем контроля и управления процессами,
                обеспечивающих защиту работающих и аварийное отключение
                производственного оборудования;
            \item своевременное получение информации о возникновении ОиВПФ на
                отдельных технологических операциях;
            \item обеспечение пожарной и взрывной безопасности процесса;
            \item выполнение требований охраны окружающей среды;
            \item другие факторы.
        \end{itemize}

        При выборе производственного помещения или промышленной площадки (для
        процессов, осуществляемых вне помещений) учитывают соответствие
        помещений и площадок требованиям строительных норм и правил, а также
        стандартам безопасности – уровни производственных факторов на рабочих
        местах не должны превышать допустимые значения. 

        Следующим мероприятием является выбор производственного оборудования,
        его размещение и организация рабочих мест. Оборудование должно
        соответствовать требованиям ГОСТ 12.2.003-91.  Размещение выбранного
        оборудования и организацию рабочих мест необходимо осуществлять с
        учетом минимизации опасных и вредных производственных факторов. 

        При организации рабочих мест руководствуются следующими принципами,
        изложенными в ГОСТ 12.2.061-81 ССБТ «Оборудование производственное.
        Общие требования безопасности к рабочим местам»: конструкция рабочего
        места, его размеры и взаимное расположение его элементов (органов
        управления, СОИ, кресел, вспомогательного оборудования и т.п.)
        соответствуют:
        \begin{itemize}
            \item антропометрическим, физиологическим и психофизиологическим
            данным человека;
            \item характеру работы.
        \end{itemize}

        Конструкция рабочего места обеспечивает:
        \begin{itemize}
            \item выполнение трудовых операций в зонах моторного поля
                (оптимальной, легкой досягаемости) в зависимости от требуемой
                точности и частоты действий (определение зоны моторного поля
                производится согласно требованиям ГОСТ 12.2.032-78 ССБТ
                «Рабочее место при выполнении работ сидя. Общие эргономические
                требования» и ГОСТ 12.2.033-78 ССБТ «Рабочее место при
                выполнении работ стоя.  Общие эргономические требования»);
            \item устойчивое положение и свободу движений работающего,
                безопасность выполнения трудовых функций, исключение или допуск
                в редких случаях кратковременной работы (например, сильно
                наклоняться вперед или в стороны, приседать, работать с
                вытянутыми или высоко поднятыми руками и т.д.), вызывающей
                повышенную утомляемость; рациональное размещение
                технологической и  организационной оснастки на рабочем месте;
            \item необходимый обзор наблюдений – СОИ (средства отображения
                информации) размещаются в зонах информационного поля рабочего
                места с учетом частоты и значимости поступающей информации;
            \item возможность управления - органы управления (ОУ) должны быть
                размещены с учетом рабочей позы, функционального назначения ОУ,
                частоты применения, последовательности использования,
                функциональной связи с соответствующими СОИ; расстояние между
                ОУ должны исключать возможность изменения положения ОУ при
                манипуляции со смежными органами;
            \item безопасность выполнения работ - достигается за счет
                выполнения комплекса мероприятий; при наличии работ, связанных
                с воздействием на работающих опасных и (или) вредных
                производственных факторов, рабочее место должно быть оснащено
                средствами защиты, средствами пожаротушения и спасательными
                средствами; удобное и безопасное обслуживание и ремонт
                оборудования.
        \end{itemize}

        Пульт управления является основным функциональным элементом рабочих
        мест с автоматизированным управлением, а также рабочих мест смешанного
        типа с наличием автоматизированного элемента.

        Пульт управления может быть конструктивно отделен от объекта управления
        и даже находиться в другом помещении, а может являться и частью
        оборудования. Возможны комбинированные варианты. При работе за пультом
        управления оператор может получать информацию с
        контрольно-измерительных приборов, расположенных на пульте, приборном
        щите, или непосредственно с управляемого объекта. В последнем случае на
        пульте располагаются только органы управления. 

        Выбор пульта и панели управления, их количество зависят от типа
        рабочего места, его назначения и организации. Габаритные размеры
        пульта, параметры зон досягаемости по высоте, ширине и глубине, размеры
        пространства для ног, высота рабочей поверхности, углы наклона панелей
        рассчитываются по общим правилам расчета параметров рабочих мест. Эти
        параметры зависят от параметров сиденья и объема рабочего пространства.

        Общие эргономические требования к пультам управления установлены ГОСТ
        23000-78. Элементы пультов управления должны удовлетворять нормативным
        требованиям. При взаимном расположении элементов рабочего места
        оператора необходимо учитывать (ГОСТ 22269-76):
        \begin{itemize}
            \item рабочую позу человека-оператора;
            \item пространство для размещения человека-оператора;
            \item возможность обзора элементов рабочего места;
            \item возможность осуществления всех необходимых движений
                и перемещений для эксплуатации и технического обслуживания
                оборудования;
            \item возможность обеспечения оптимального режима труда и отдыха.
        \end{itemize}

        Средства отображения информации, устанавливаемые на пультах управления,
        соответствуют требованиям ГОСТ 22902-78 и ГОСТ 21829-76. Средства звуковой
        сигнализации соответствуют требованиям ГОСТ 21786-76. Мнемосхемы
        соответствуют требованиям ГОСТ 21480-76. Органы управления соответствуют
        требованиям ГОСТ 21752-76, ГОСТ 21753-76, ГОСТ 22613-77, ГОСТ 22614-77,
        ГОСТ 22615-77. Способы кодирования зрительной информации соответствуют
        требованиям ГОСТ 21829-76.

        Освещенность в рабочей зоне устанавливается СНиП II-4-79 и отраслевыми
        нормами соответствующих производств.

    \subsection{Требования к электробезопасности}

        Основными причинами электротравматизма являются:
        \begin{itemize}
            \item нарушение  правил устройства,  технической  эксплуатации и
                техники безопасности электроустановок;
            \item неисправность изоляции, из-за чего металлические
                нетоковедущие части оборудования оказываются под напряжением;
            \item обрыв заземляющего проводника;
            \item использование электрозащитных устройств, не отвечающих
                условиям выполнения работ;
            \item выполнение электромонтажных и ремонтных работ под напряжением;
            \item низкое качество соединений и ремонта;
            \item недооценка  необходимости  выключения электроустановки
                (снятия напряжения) в нерабочие периоды;
            \item невыполнение периодических испытаний, в частности проверок
                сопротивления изоляции и сопротивлений заземляющих устройств;
            \item отсутствие маркировки, предохранительных плакатов,
                блокировок, временных ограждений мест электротехнических работ.
            \item отсутствие контроля за действиями работников со стороны ИТР
                или исполнителей работ.
        \end{itemize}

        Все мероприятия по профилактике электротравматиза можно разделить на
        организационные и технические.

        К организационным мерам относятся: нормативные документы, разделение
        сетей и помещений по степени опасности поражения электрическим током,
        разделение персонала на квалификационные группы, обучение, инструктаж,
        соответствующая организация работ, медосмотры и т.п.
    
        Основные нормативные документы по электробезопасности:
        \begin{itemize}
            \item «Правила устройства электроустановок» (ПУЭ);
            \item «Правила технической эксплуатации электроустановок
                потребителей» (ПТЭ);
            \item «Правила техники безопасности при эксплуатации
                электроустановок потребителей» (ПТБ).
        \end{itemize}

        Согласно ПУЭ электрические сети питающие относятся к сетям до 1000 В.

        В соответствии с ПУЭ помещения в которых расположены электродвигатели
        по степени опасности поражения электрическим током относятся к
        помещениям с повышенной опасностью (есть один признак повышенной
        опасности); 

        Технические мероприятия по профилактике электротравматиза можно
        разделить на три группы: 
        \begin{itemize}
            \item мероприятия, осуществляемые при нормальном режиме работы
                электроустановок;
            \item мероприятия, осуществляемые при аварийном режиме работы
                электроустановок;
            \item применение системы электрозащитных средств.
        \end{itemize}

        К электрозащитным средствам относятся:
        \begin{itemize}
            \item изоляция токоведущих частей;
            \item оградительные устройства (ограждения);
            \item применение малых напряжений;
            \item ограждающие средства;
        \end{itemize}

        Для обеспечения нормальной работы электроустановок и защиты от
        поражения электрическим током применяется рабочая изоляция –
        электрическая изоляция токоведущих частей. Может предусматриваться
        также дополнительная изоляция для защиты в случае повреждения рабочей
        изоляции. Изоляция, состоящая из рабочей и дополнительной, называется
        двойной изоляцией.

    \subsubsection{Оградительные устройства (ограждения)}

        С целью исключения возможности прикосновения к токоведущим частям или
        приближения к ним на опасное расстояние применяются ограждения.
        Защитные ограждения должны обладать соответствующими электрическими и
        механическими свойствами. Они могут иметь различное конструктивное
        исполнение (сплошные, сетчатые). Ограждения должны сниматься или
        открываться специальным инструментом или ключом.

    \subsubsection{Применение малых напряжений}

        Малым называется номинальное напряжение не более 42 В, применяемое в
        целях уменьшения опасности поражения электрическим током. Малые
        напряжения используются для питания электрифицированного инструмента,
        переносных светильников и местного освещения в помещениях с повышенной
        опасностью и особо опасных.

        Ограждающие средства служат для временного ограждения токоведущих
        частей, а также предупреждения ошибочных операций с коммутационной
        аппаратурой. К ним относятся переносные ограждения (щиты, клетки),
        изолирующие накладки, переносные заземления.

        Защитное заземление – это преднамеренное электрическое соединение с
        землей или с ее эквивалентом металлических нетоковедущих частей,
        которые могут оказаться под напряжением. Назначение защитного
        заземления – устранение опасности поражения людей электрическим током
        при появлении напряжения на конструктивных частях электрооборудования,
        то есть при замыкании на корпус.  Принцип действия защитного заземления
        – снижение до безопасных значений напряжений прикосновения и шага,
        обусловленных замыканием  на  корпус.  Это достигается  снижением
        потенциала заземленного оборудования, а также выравниванием потенциалов
        за счет поднимания потенциала основы, на которой стоит человек, к
        потенциалу, близкому по значению  потенциалу заземленного оборудования.

        Область применения защитного заземления – трехфазные сети напряжением
        до  1000 B с любым режимом нейтрали.

        Заземляющее устройство – это совокупность конструктивно объединенных
        заземляющих проводников и заземлителя.

        Заземляющий проводник – это проводник, который соединяет заземляемые
        объекты с заземлителем.  Если заземляющий проводник имеет два или
        больше ответвлений, то он называется магистралью заземления.

        Заземлитель – это совокупность объединенных проводников, которые
        находятся в контакте с землей или с ее эквивалентом. Различают
        заземлители искусственные, предназначенные исключительно для
        заземления, и естественные металлические предметы, которые находятся в
        земле.

        B качестве искусственных заземлителей применяют вертикальные и
        горизонтальные электроды.

        B качестве естественных заземлителей можно использовать:
        \begin{itemize}
            \item проложенные в земле водопроводные и другие металлические
                трубопроводы, за исключением трубопроводов горючих жидкостей,
                горючих или взрывоопасных газов, а также трубопроводов,
                покрытых изоляцией для защиты от коррозии;
            \item обсадные трубы артезианских колодцев, скважин, шурфов;
            \item металлические конструкции  и арматуру железобетонных
                элементов зданий и сооружений, которые соединены с землей;
            \item свинцовые оболочки кабелей, проложенных в земле.
        \end{itemize}

        Естественные заземлители имеют преимущественно малое сопротивление
        растеканию тока, поэтому использование их в качестве заземлителей
        позволяет экономить значительные средства. Недостатком естественных
        заземлителей является доступность их не электротехническому персоналу и
        возможность нарушения непрерывности соединения протяженных
        заземлителей. B качестве заземляющих проводников, предназначенных для
        соединения заземляемых частей с заземлителем, применяют ленточную и
        круглую сталь. Заземляющие проводники прокладывают открыто по
        конструкциям здания, в том числе по стенам на специальных опорах.
        Заземляемое оборудование присоединяют к магистрали заземления при
        помощи отдельных проводников. При этом последовательное заземление
        оборудования не допускается.

        Защитному заземлению подлежат металлические нетоковедущие части
        оборудования, которые из-за неисправности изоляции могут оказаться под
        напряжением и к которым возможно прикосновение людей или животных. При
        этом в помещениях с повышенной опасностью и в особо опасных по условиям
        поражения током, а также во внешних установках заземление обязательно
        при номинальном напряжении электроустановки более 42 B переменного и
        более 110 B постоянного тока, а в помещениях без повышенной опасности -
        при напряжении 380 B и выше переменного тока; 440 B и выше постоянного
        тока. Только во взрывоопасных помещениях заземление выполняется
        независимо от значения напряжения установки.

        Заземлению не подлежат корпуса электрооборудования, аппаратов и
        электромонтажных конструкций, установленные на заземленных
        металлических конструкциях, распределительных устройствах, в щитах,
        шкафах, на станинах станков, машин и механизмов, при условии надежного
        электрического контакта с заземленным основанием, арматура изоляторов
        всех типов, растяжки, кронштейны и осветительная арматура при установке
        их на деревянных опорах воздушных линий электропередач или на
        деревянных конструкциях открытых подстанций.

    \subsection{Требования к пожаробезопасности}

        Анализ причин пожаров, происходящих на промышленных предприятиях
        свидетельствует о том, что одной из основных причин их возникновения
        является неисправность и неправильность эксплуатации электротехнических
        установок и устройств. В большинстве случаев пожары происходят из-за
        коротких замыканий в электрических сетях, перегрева и воспламенения
        веществ и материалов, находящихся в непосредственной близости от
        электрооборудования, токовых перегрузок проводов и оборудования,
        больших переходных сопротивлений, электрических искр и др.

        При проектировании электрических установок, прежде всего, необходимо,
        чтобы все оборудование по своему исполнению соответствовало характеру
        окружающей среды и технологическому процессу.

        Определение категорий помещений по взрывопожарной и пожарной опасности
        производится по методике, приведенной в Общесоюзных нормах
        технологического проектирования (ОНТП 24-86).

        Основной профилактической мерой относительно предупреждения пожаров и
        взрывов от электрооборудования является правильный выбор и эксплуатация
        такого оборудования во взрыво- и пожароопасных помещениях: в
        соответствии с Правилами устройства электроустановок (ПУЭ) помещения
        подразделяются на взрывоопасные и пожароопасные  зоны.

        Взрывоопасная зона – это пространство, в котором есть или могут
        образовываться взрывоопасные смеси.

        Пожароопасная зона - это пространство, где могут находиться горючие
        вещества как при нормальном технологическом процессе, так и при
        возможных его нарушениях.

        B зависимости от класса взрывоопасных и пожароопасных зон производится
        выбор электрооборудования, устанавливаемого в этих зонах.

        B соответствии с ПУЭ в пожароопасных зонах устанавливается
        электрооборудование закрытого типа, внутреннее пространство которого
        отделено от внешней среды оболочкой. Аппаратуру управления и защиты,
        светильники рекомендуется применять в пыленепроницаемом исполнении. Вся
        электропроводка должна иметь надежную изоляцию.

        Bo взрывоопасных зонах следует устанавливать взрывозащищенное
        оборудование, изготовленное в соответствии с ГОСТ 12.2.020-76. Пусковую
        аппаратуру, магнитные пускатели для классов B-I и B-II необходимо
        выносить за пределы взрывоопасных зон. Проводка во взрывоопасных
        помещениях должна прокладываться в металлических трубах. Может
        использоваться бронированный кабель. Светильники для классов B-I, B-II,
        B-IIa должны иметь взрывозащищенное исполнение.

